\documentclass[../MH_Total.tex]{subfiles}
\newcommand{\Dd}{\mathcal{D}}
\begin{document}

Let $\Dd$

Let's get excited for day three of the lecture.

The afternoon will consist only of one hour. IT's logic day! One lecture about reverse mathematics at 15:30. After that there's a lecture of Peter's. 

In usual mathematics you use axioms you deduce theorems. In reverse mathematics you try to find the axioms which make a statement true.

I didn't say much about homotopy limits, and somebody asked me right after the lecture, so let's do this first.

\subsection{Homotopy Limits}
Remember how we constructed the homotopy pushout: we had factorizations
\begin{center}
	\begin{tikzcd}
		A \arrow[r] \arrow[d] & Mf \arrow[r,"\sim"] \arrow[rdd, bend left] \arrow[d] & B\\
		Mg \arrow[d,"\sim"]  \arrow[drr, bend right] \arrow[r] & P \arrow[dr,dotted,"\exists!"]\\
		C & & Z
	\end{tikzcd}
\end{center}
and we call $P$ the homotopy pushout. It has the universal property that
\begin{align*}
\mathbf{Top}(P,Z) &= \{(r,s,h) \,|\, r: B \to Z, s: C \to Z, h: A \times I \to Z, h|_{A \times \{0\}} = r \circ f, h|_{A \times \{1\}} = s \circ f\}\\
	k &\mapsto (k_C,k_B,k_{A \times I})
\end{align*}

So not coincidence but homotopy between the two things. 

Homotopy \emph{limits}: we can factorize any map by a weak equivalence followed by a fibration. 

\begin{center}
	\begin{tikzcd}
		Y^I  = \{f: I \to Y \,|\, f(0) = p\} \cong * \arrow[d]\\
		Y \ni p
	\end{tikzcd}
\end{center}
mapping $f$ to $f(1)$. Homotopic to a point. This right here is a fibration. If we have a homotopy downstairs, you can use this interval space here to lift it up somehow.

The recipe for homotopy limit is then:

special case:
\begin{center}
	\begin{tikzcd}
		\{f: I \to Y \,|\, f(1) = f(0) = p\} = \Omega Y \arrow[d] & * \sim Y^I\arrow[d,"p"]\\
		* \arrow[r,"p"] & Y
	\end{tikzcd}
\end{center}
One can show that pullbacks of fibrations are fibrations again - the lifting will exist by the universal property of the pullback. 

We can write a section $X \to X \times Y^I, x \mapsto x,C_{f(x)}$. $Y^I$ is contractible, so you can imagine that $X \to X \times Y^I$ is a homotopy. So you factor $f: X \to Y$ as
\begin{center}
\begin{tikzcd}	
	X \times Y^I \arrow[r,"\op{pr}"] \arrow[d] & Y^I \arrow[d,two heads]\\
	X \arrow[r,"f"] \arrow[r,bend left,"\sim"] & Y
\end{tikzcd}
\end{center}
Also leads us to a point of view on this phenomenon.

\section{$(\infty, 1)$-categories}
Given a relative category $(\Cc,W)$ the first thing we did was to localize to obtain $\Cc \xrightarrow{L} \Cc[W^{-1}]$. Then we had to take some right Kan extension, etc. It's not intrinsic somehow, you have to remember the functor $L$ and so the particular presentation $\Cc$. There can be several such pairs of categories and $W$'s which give you the same thing, as we saw last week. 

We're going to do this in a more refined way now. We'd like to give another universal property of this homotopy pushout: we just said that 
\begin{align*}
\mathbf{Top}(P,Z) &= \{(r,s,h)\,|\,\dots\}\\
&\cong \mathbf{Top}(B,Z) \times^h_{\op{Top}(A,Z)} \mathbf{Top}(C,Z)
\end{align*}
(Homotopy fiber product) We can give a topology on this space here of maps $P \to Z$, and we can make this into an equivalence of topological spaces.

\begin{exercise}
 	Check this. You need to know there is a space such that $\mathbf{Top}(A \times B,C) \cong \mathbf{Top}(A,C^B)$. You should just use that there are such exponential objects in these spaces.
 \end{exercise} 
Somehow this universal property is determined by $A,B,C$ by equivalence, and you can phrase it in this way through an equivalence to another space. 

So: the universal property of this homotopy pushout can be phrased as a sort of pullback. That is, the universal property of the pushout is that mapping out of it turns it into a homotopy pullback.

Compare this to Set-based Category theory. In category theory, you define limits and colimits by mapping properties. You can rephrase the usual pushout of $A \to B,C$ as
\[
\Cc(P,Z) \cong \Cc(B,Z) \times_{\Cc(A,Z)} \Cc(C,Z). 
\]
So to define a pushout in $\mathbf{Set}$ it's enough to define pushouts in any category $\Cc$ (this is a pushout in sets). 

We know what homotopy limits in topological spaces are. Now we can say in these terms what it is in any relative category, in any such localization $\Cc \to \Cc[W^{-1}]$. 

\begin{definition}[(Hammock Localization)]
	Another way of inverting arrows. Except you don't end up in just another category but in acategory enriched in simplicial sets.

	To a relative category $(\Cc,W)$ with $(\op{id} \subseteq W$, closed under composition) you attach the category enriched in simplicial sets $L^H_W(\Cc)$ given by
	\begin{align*}
		\op{Ob}(L^H_W(\Cc)) &:= \op{Ob}(\Cc)\\
		L_H^W(\Cc)(A,B) &:= \text{simplicial set with}
	\end{align*}
	\[
	L_W^H(\Cc)(A,B)_n = (*)
	\]
	for $k \in \N$. This diagram should be commutative, the vertical arrows are $\in W$, wrong way arrows are always $\in W$, and in every column all horizontal arrows should point in the same direction. Then we mod out by the equivalence relation $\sim$ generated by
	\begin{enumerate}
		\item Deleting identity columns;
		\item Composing columns pointing in the same direction;
		\item We \emph{don't} give the cancellation properties. 
	\end{enumerate}
	($n$ is the width of the hammock.)
	The face maps are given by composing vertical arrows, thereby forgetting one of these rows.
	The degeneracy maps are given by copying one row and adding vertical identity arrows.
	(It's kind of like a nerve, and the reason you can do this is because the nerve is always a simplicial set.)
	Composition of morphisms is given by stringing hammocks together (horizontally). 
\end{definition}

With this we can define homotopy limits and colimits more intrinsically. Let $(\Cc,W)$ be given. Then $L_W^H(\Cc)$ has a universal property with respect to 

\begin{example}
	Let $P$ be the hopushout of a diagram
	\begin{center}
		\begin{tikzcd}
			A \arrow[r] \arrow[d] & B\\
			C
		\end{tikzcd}
		$\in \Cc$
	\end{center}

	Then the universal property is $L_H^W(\Cc)(P,Z) \cong L_W^H(\Cc)(B,Z) \times^h_{\Cc(A,Z)} L_W^H(\Cc)(C,Z)$, where $\cong$ is a weak equivalence in simplicial sets. The latter is a homotopy pullback in simplicial sets (or in $\mathbf{Top}$ if one prefers, take geometric realizations), and the former is a simplicial set by definition.
\end{example}

Is this really necessary? It seems to be a monstrosity. What are the advantages? It's an intrinsic object upto equivalence of $\infty$ categories. We had pairs of different categories which gave different presentations of the same thing, but they give equivalence $L_H$'s.

\begin{definition}
	Given a simplicially enriched category we can define its \emph{homotopy category} $Ho\Cc$ which is a category with $\op{Ob}Ho(\Cc) := \op{Ob}\Cc$ and $Ho(\Cc)(A,B) := \pi_0(\Cc(A,B)_\bullet) = \pi_0(|\Cc(A,B)_\bullet|)$.
\end{definition}

\begin{proposition}
	$Ho(L_W^H(\Cc)) = \Cc[W^{-1}]$.
\end{proposition}

\begin{proof}
	We had two of the three properties of the localization in the definition of the Hammock localization. We just need to show the third property holds (**). It's a 1-simplex between the two rows, so they're identified. Let's call that a proof.

	[A path is a zig-zag between 1-simplices.] Upto the relation of lying in the same connected components, such a pair of wrong way and right way $f$s are connected to the identity path. They become the same representatives of a morphism in the homotopy category.
\end{proof}

\marginpar{Peter is naming these 1 1/2, 1 3/4, etc. Ingo: ``It's a good thing that the rationals are dense in the reals.''}
\begin{definition}
	Let $\Cc$ be a simplicial category. $A,B \in \op{Ob}\Cc$ are equal if they are isomorphic in $Ho(\Cc)$. 
\end{definition}

\begin{definition}
	Let $\Cc$, $\mathcal{D}$ be simplicially enriched categories and $\Cc \xrightarrow{f} \mathcal{D}$ be a enriched functor, i.e., 
	\begin{align*}
		\Cc(A,B)_\bullet \to \mathcal{D}(F(A),F(B))_\bullet
	\end{align*}
	is a map of simplicial sets. We call this a (Dwyer-Kan) equivalence if 
	\begin{enumerate}
		\item Every object in $\mathcal{D}$ is equivalence to one in $F(\Cc)$. ``Essentially surjective''
		\item All of the maps $\C(A,B)_\bullet \to \mathcal{D}(F(A),F(B))_\bullet$ are weak equivalences, which we think of as ``fully faithful'' in our new setting. 
	\end{enumerate}
\end{definition}
This is what it means to be the same presentation. 

An $(\infty,1)$-category is a simplicially enriched categories up to equivalence. If you take a space and you take the points as objects, and the paths as morphisms, homotopies between paths as a 2-morphism, and homotopies between homotopies as 3-morphisms, etc. you have a whole tower of morphisms at every level. This is the basic example of an $\infty$-category. 

Higher simplices are the higher morphisms.

We have the pair

(Simplicially enriched categories with enriched functors, DK-equivalences)

And this is a relative category, a model for $(\infty,1)$-category. There are different ``models'' for $(\infty,1)$-categories, e.g. quasicategories (Borchan-Vogt, Joyal, Lurie). 

We wanted to invert some morphisms in a category, got a natural notion of equivalence. Now we get asimplicially enriched category from this sort of thing. But why do we need to develop a theory of simplicially enriched categories of $(\infty,1)$-categories?
\begin{theorem}[(Dwyer-Kan)]
	Every simplicially enriched category $\Cc$ is Dwyer-Kan equivalent to some $L_W^H(\mathcal{D})$ for some class of morphisms. 
\end{theorem}
(This is completely general $\infty$-category theory.) So we didn't get anything that we didn't want.

To gloss over this derived functor business and to replace spaces by other spaces, etc. we're just going to assume everything is derived. In $(\infty,1)$-categories say colimit, limit, Kan extension, etc. for what has the usual universal property, but up to weak equivalence of the hom simplicial sets.

For example, $X \in \Cc$ a simplicially enriched category, write $\Sigma X$ for the pushout of $X$ along two points
\begin{center}
	\begin{tikzcd}
		X \arrow[r] \arrow[d] &*\\
		*
	\end{tikzcd}
\end{center}
because that's what we'd get in spaces if we'd derived. We replace points by mapping cylinders, then you glue and obtain a suspension (***). Or e.g. $X$ is a ``terminal object'' if $\Cc(Z,X) \cong *$ for all $Z \in \Cc$ up to equivalence (if it's an isomorphism it's actually a terminal object).

The time has come to remind us of this hexagon. Going infinity-categorical, we write a more refined localization. 
$\mathbf{Spaces}$ play the role of Sets. 

The next thing we have is stabilizaiton. Cohomology theories have this suspension isomorphism $H^n(\Sigma X) \cong H^{n-1}(X)$. So $\Sigma$ should factorize through somewhere where suspension becomes an equivalence, the morphism $\mathbf{Spaces}_* \xrightarrow{\Sigma^\infty} \mathbf{Spaces}_*\langle \Sigma^{-1} \rangle$. Then we go to $H\Z\op{-Mod} = D(Ab)$. 

\section{Stabilization}
As we go around the hexagon the categories become more and more algebraic [The homotopy category of Spectra is additive?]

What we want: given an $(\infty-)$category $(\C,F: \C \to \C)$ we want a category $(\C\langle F^{-1} \rangle, \overline{F} \xrightarrow{\sim} \overline{F})$ such that for any $H: (\Cc,F) \to (\mathcal{D},\mathcal{D} \xrightarrow{G} \mathcal{D})$ which commutes with these endofunctors, i.e. $H \circ F \cong G \circ H$, the functor $H$ should factorize through $\C\langle F^{-1} \rangle$ (5*). 

A morphism of infinity categories with endofunctors $(\Cc,F) \to (\mathcal{D},G)$ is a functor such that $H \circ F = G \circ H$. 

How to construct this? Look at modules. If you have a module $M$ over a ring $R$, we can localize the module $M$ at some point $r \in R$ to obtain $M[r^{-1}]$, which we can think of as
\[
M[r^{-1}] = \op{colim}(M \xrightarrow{\cdot r} M \xrightarrow{\cdot r} M \to \dots)
\]
(6*)

Same strategy for stabilization:
\[
\Cc \to \op{colim}(\Cc \xrightarrow{F} \Cc \xrightarrow{F} \Cc \to \dots) =: \Cc \langle F^{-1} \rangle.
\]
(colimit in our infinity category of infinity categories)

\begin{exercise}
	Find an example in $\mathbf{Cat}$ (so a normal category) and see whether one can understand it. 
\end{exercise}

\begin{example}
	The category $\mathbf{Spaces}_*$ with the suspension endofunctor $\Sigma$. The category $\mathbf{Spaces}_*$ is very nice: complete, cocomplete, etc. It's presentable, i.e., every space is a directed colimit of spheres and the one-point space $*$. This is the statement that every topological space is weak equivalent to a CW-complex; disks are the points, since disks are contractible.

	Call $\op{Pres}^L := \infty$-category of presentable $\infty$-categories and functors which are left adjoints ($\Leftrightarrow$ colimit-preserving functors). If we consider right adjoints, we get $\op{Pres}^R$, but $\op{Pres}^L = (\op{Pres}^R)^{op}$. One great place to read about this is Zizinski's Bourbaki article. It's a new syntax: you use the same words, but they mean different things now. It turns out that virtually everything in usual category theory is true in $\infty$-categories, though it's sometimes monstrous to prove.

	An adunction in $(\mathbf{Top}_*/\mathbf{Spaces}_*$: 7* ($\Omega Y$: paths which start at the base point). This adjunction passes over to $\mathbf{Spaces}_*$.

	By definition stabilization inverts the suspension functor:
	\[
	\mathbf{Spaces}_*\langle \Sigma^{-1} \rangle := \op{colim}_{\text{in }Pres^L} (\mathbf{Spaces}_* \xrightarrow{\Sigma} \mathbf{Spaces}_* \xrightarrow{\Sigma} \dots)
	\]
	Colimits become limits in $\op{Pres}^R$, and the equivalence comes from taking adjoints, so 
	\[
	\mathbf{Spaces}_* = \op{lim}_{\text{in }\op{Pres}^R} (\mathbf{Spaces}_* \xleftarrow{\Omega} \mathbf{Spaces}_* \xleftarrow{\Omega} \mathbf{Spaces}_* \xleftarrow{\Omega} \dots)
	\]
	Of course you have to know that you can take such a colimit, and it's a fact by Lurie that $\op{Pres}^L$ is cocomplete.

	In usual category theory we take a product of all of these objects and take a subobject of compatible objects. If we have $X_0,X_1,X_2,\dots$ in these spaces, the objects of this limit are
	\[
	(X_i, f: \Omega X_{i+1} \xrightarrow{\sim} X_i \,|\, i \in \N). 
	\]
	So the weak equivalences $f$ are part of the datum now.

	Brown representability theorem: if a functor $F: \op{Ho}\mathbf{Top}_*^{op} \to \mathbf{Set}$ satisfies
	\begin{enumerate}
		\item $F(\bigvee_\alpha X_\alpha) = \prod_\alpha F(X_\alpha)$;
		\item (8*)
	\end{enumerate}
	Then there exists $Y \in \mathbf{Top}_*$ such that $F(-) \cong \op{Ho}(\mathbf{Top}_*)(-,Y)$. So the theorem gives a condition for representability. In particular, if $E^*$ is a cohomology theory then each $E^n$ satisfies (1) and (2) (Exercise! Something like Mayer-Vietoris for (2)) so each $E^n$ is representable, i.e. we have objects $E_n$ such that
	\[
	E^n(-) = \op{Ho}(\mathbf{Top}_*)(-,E_n). 
	\]
	We have suspension isomorphisms
	\[
	\op{Ho}(\mathbf{Top}_*)(\Sigma X,E_n) \cong E^n(\Sigma X) \cong E_{n-1}(X) \cong \op{Ho}(\mathbf{Top}_*)X,E_{n-1}). 
	\]
	But $\op{Ho}(\mathbf{Top}_*)(\Sigma X,E_n) \cong \op{Ho}(\mathbf{Top}_*)(X,\Omega E_n)$ so by the Yoneda lemma we get $\Omega E_n \cong E_{n-1}$. This precisely means that we have a spectrum (an object of $\mathbf{Spaces}_*\langle \Sigma^{-1} \rangle$) representing $E^*$.
\end{example}

% %-------------------------------------------------
% Proposition 1.3.7
%
%-------------------------------------------------

\documentclass[tikz]{standalone}

\usetikzlibrary{cd}
\usetikzlibrary{arrows}
\usepackage{amsfonts,amssymb}
\usepackage{mathtools}
\usepackage{adjustbox}

\def\zigzagU{
	\adjustbox{scale=0.5}{%
		\tikz[baseline=.1em,scale=0.5,node distance=6em]{
		        \node(1){$X$};
		        \node(2)[right of=1]{$X_1$};
		        \node(3)[right of=2]{$X_2$};
		        \node(4)[right of=3]{$Y$};
		        \draw[->](2) to node [above] {$\sim$} node [below] {$f$} (1);
		        \draw[->](2) to node [below] {$g$} (3);
		        \draw[->](4) to node [above] {$\sim$} node [below] {$h$} (3);
		}
	}
}
\def\zigzagD{
	\adjustbox{scale=0.5}{%
		\tikz[baseline=.2em,scale=0.5,node distance=6em]{
		        \node(1){$F(X)$};
		        \node(2)[right of=1]{$F(X_1)$};
		        \node(3)[right of=2]{$F(X_2)$};
		        \node(4)[right of=3]{$F(Y)$};
		        \draw[->](1) to node [above] {$F(f)^{-1}$} (2);
		        \draw[->](2) to node [above] {$F(g)$} (3);
		        \draw[->](3) to node [above] {$F(h)^{-1}$} (4);
		}
	}
}


\begin{document}
	\begin{tikzcd}
		\mathcal{C} \ar[r] \ar[dr,"F"', bend right] & \mathcal{C}[W^{-1}] \ar[d,dashed,"\bar{F}"] \ar[r,"\ni",phantom]& (\zigzagU) \ar[d,mapsto]
		\\
		& \mathcal{D} \ar[r,"\ni",phantom] & (\zigzagD)
		\\
		F(W)\subseteq \text{Iso}(\mathcal{D})
	\end{tikzcd}
\end{document}
\end{document}