\documentclass[../MH_Total.tex]{subfiles}
\begin{document}
\section{Interlude: Simplicial Sets}
\begin{definition}
	$\Delta$ is the category of (nonempty) finite linearly ordered sets with order-preserving morphisms. So its objects consist of the ordered set $[n] = \{1 < 2 < \dots < n\}$ for each $n \in \Z^{\ge 0}$.
\end{definition}

\begin{example}
	Some particular morphisms:
	\begin{align*}
		\delta_i: [n] &\to [n+1]\\
		j &\mapsto
		\begin{cases}
			j & \text{ if } j < i\\
			j+1 & \text{ if } j \ge i
		\end{cases}
	\end{align*}
	That is, $\delta_i$ omits $i$.

	We have map going the other way around:
	\begin{align*}
		\sigma_i: [n+1] &\to [n]\\
		j &\mapsto
		\begin{cases}
			j &\text{ if } j \le i\\
			j-1 &\text{ if } j > i
		\end{cases}
	\end{align*}
	So you merge $i$ into $i+1$ and leave everything else the same. [This isn't quite right - you pull back everything starting from $i+1$, hitting $i$ twice.]
	% $\sigma_i: [n+1] \to [n]$, $\sigma_i(i) = \sigma_i(i+1)$ and $\sigma_i(j) = j$ otherwise. So you merge two of them and leave everything else the same.
\end{example}

\begin{remark}\label{simplicial_composition}
Every morphism in $\Delta$ is the composition of such morphisms.
There are relations between these morphisms, but we won't mention them here. 
\end{remark}

\begin{definition}
	A \emph{simplicial set} is a functor $X: \Delta^{op} \to \mathbf{Set}$, say $[n] \mapsto X_n$. 
\end{definition}

We can specify such a functor by the images of its objects and the $\delta_i$ and $\sigma_i$, since by Remark \ref{simplicial_composition} these two classes of maps generate all of the morphisms. This can be drawn as a diagram $X_0, X_1,X_2,X_3,\dots...$ with the morphisms between them:
\begin{center}
	\begin{tikzcd}
		X_0 \arrow[from=r, shift right = 5,swap,"X(\delta_0)"] \arrow[r,"X(\sigma_0)"] \arrow[from=r, shift left = 5,swap,"X(\delta_1)"] & X_1  \arrow[from=r,shift right = 4] \arrow[from=r,shift right = 0] \arrow[from=r,shift left = 4] \arrow[r, shift right = 2] \arrow[r, shift left = 2]& X_2 \cdots
	\end{tikzcd}
\end{center}

[The category $\Delta$ has some relations between the face and boundary maps. Any collection of sets together with appropriate arrows between them which satisfy (the opposite of) these relations is a simplicial set.]

\begin{definition}
	The \emph{$n$-simplex} $\Delta^n$ is defined to be
	\[
	\Delta^n := \{(x_0,\dots,x_n) \in \R^{n+1}\,|\, 0 \le x_i \le 1, \sum x_i = 1\}.
	\]
	[The equation defines a plane, and the inequalities bound the plane.]
\end{definition}

We think of the maps in the simplicial set as gluing data. It's a blueprint for gluing a space.

More formally:
\begin{definition}
	Given a simplicial set $X_\bullet$ its \emph{geometric realization} is defined to be
	\[
	|X_\bullet| := \left(\prod_{n \in \N_0} X_n \times \Delta^n\right)/\sim = \int^n X_n \times \Delta^n.
	\]
	where $\sim$ identifies simplicies as prescribed by the ``face maps''. (It's a special kind of colimit.) We can define the geometric realization of a morphism between simplicial sets and so we get a functor $|-|: \mathbf{SimpSet} \to \mathbf{Top}$. 
\end{definition}

It's a combinatorial blueprint for making topological spaces. It's okay not to think of the degeneracy maps if you're imagining it as a way of constructing topological spaces.

There's also a natural functor going the other way:

\begin{definition}
	Let $Y$ be a topological space. Then we get a simplicial set 
	\[
	\op{Sing}(Y)_n := \{\Delta^n \to Y \text{ continuous}\}
	\]
	where the face maps $\op{Sing}(Y)_n \to \op{Sing}(Y)_{n-1} = \{\Delta^{n-1} \to Y\}$ are given by the restriction map to faces. This defines a functor $\op{Sing}(-): \mathbf{Top} \to \mathbf{SimpSet}$. 

	[More precisely, I think, $\op{Sing}(Y)$ is the composition
	\begin{alignat*}{3}
		\Delta &\mapsto \mathbf{Top} &&\mapsto \mathbf{Set}\\
		n &\mapsto \Delta^n &&\mapsto \mathbf{Top}(\Delta^n,Y)
	\end{alignat*}
	We just needs to define the functor $n \mapsto \Delta^n$ on maps.]
\end{definition}

\begin{remark}
	From this you construct a (co)complex 
	\[
	\op{Sing}(Y)_\bullet = \left(\Z^{\op{Sing}_0(Y)} \xleftarrow{d_0 - d_1} \Z^{\op{Sing}_1(Y)} \xleftarrow{d_0 - d_1 + d_2} \Z^{\op{Sing}_2(Y)} \leftarrow \dots\right)
	\]
	whose components are the free abelian group generated by $\op{Sing}(Y)_n$. Your differentials are the alternating sum of the degeneracy maps. Singular homology is defined to be the homology of this (co)complex.

	To get cohomology, you apply the contravariant functor $\op{Hom}(-,\Z)$ to $\op{Sing}(Y)_\bullet$ to obtain a complex then take cohomology.
\end{remark}

\emph{Fact}: The geometric realization functor $|-|$ is left-adjoint to $\op{Sing}(-)$. In addition the counit $|\op{Sing}(Y)_\bullet| \to Y$ is a weak-equivalence, i.e., it induces an isomorphism on the higher homotopy groups.
Explicitly
\[
|\op{Sing}(Y)_\bullet| = \left(\bigsqcup_n \{\Delta^n \xrightarrow{f} Y\} \times \Delta^n\right)/\sim \to Y
\]
is given by $\overline{(f,x)} \mapsto f(x)$.

In particular, every topological space is weak-equivalent to a CW-complex since the geometric realization always outputs a CW-complex. 

\emph{Fact}: If $X_\bullet \in \mathbf{Set}^{\Delta^{op}}$ is a simplicial set, then the unit $X_\bullet \to \op{Sing}(|X_\bullet|)$ is also a weak equivalence, if one defines a weak equivalence $X_\bullet \xrightarrow{f} X_\bullet'$ to be a weak equivalence $|X_\bullet| \xrightarrow{|f|} |X_\bullet'|$. 

This is a shadow of the fact that these two things define the same homotopy theories. There's a more refined statement about these that we'll come to. Actually: $\op{Top}[\text{weak eq.s}^{-1}] \cong \mathbf{Set}^{\Delta^{op}}[\text{weak eq.s}^{-1}]$ is an equivalence of categories. We can deduce this now, but maybe we can do it better after this lecture.

This is an example of how different relative categories can give the same categories after inversion, i.e., the ``same'' localizations. 

That's why we can always use simplicial sets of instead of topological spaces - when we're working with weak homotopy, at least, we can use either category. The category of simplicial sets is much nicer, and in the category of topological spaces you have to do more work.

\section{Derived Functors}
Previously we had the general situation of a relative category $(\Cc,W)$, and we now know how to associate to it a localization $\Cc \xrightarrow{L} \Cc[W^{-1}]$. In this category the notion of equivalence became an isomorphism, in the best way possible.

How do we work in $\Cc[W^{-1}]$? The disappointment is that you can't work in there alone. For example, limits and colimits don't really work here, and there's a result that if you can take limits and colimits you were in the `trivial' situation in which $\Cc$ is a reflexive subcategory.

But usually we come from a category which is not so bad.

Given two localizations
\begin{center}
	\begin{tikzcd}
		\Cc \arrow[r,"F"] \arrow[d,"L"] & \mathcal{D} \arrow[d,"L'"]\\
		\Cc[W^{-1}] \arrow[r,dotted] & \mathcal{D}[V^{-1}].
	\end{tikzcd}
\end{center}
this dotted arrow exists (by the universal property) if $L'(F(W)) \subseteq \op{Iso}\mathcal{D}[V^{-1}]$. But the functor $F$ doesn't know anything about $W$ and $V$ so we can't expect this to happen in general.

\begin{example}[(Colimits)]
	Let $\Cc = \op{Top}$. Pushout: glue along their boundary. We definitely want pushouts, gluing things along their boundaries, etc. Unfortunately the pushout functor does not respect equivalences - see video for a nice example.
\end{example}

\begin{example}[(Limits)]
	See video.
\end{example}

\begin{example}
	Let $\Cc = \mathbf{Cat}$. Consider a category as $s,t: \op{Mor} \to \op{Ob}$, which specifies the source and target of any morphism. It's cool but long.
\end{example}

Three examples of failure. What do we do about it? The remedy is \emph{derived functors}. 

Suppose we have a diagram
\begin{center}
	\begin{tikzcd}
		\Cc \arrow[r,"F"] \arrow[d,"L"] & \mathcal{D} \arrow[d,"L'"]\\
		\Cc[W^{-1}] & \mathcal{D}[V^{-1}].
	\end{tikzcd}
\end{center}
as above. The answer to our problem is the notion of a Kan extensions, more specifically the left/right Kan extension of $L' \circ F$ along $L$. It's not obvious they exist, and they won't in general. [\textbf{See video}]. 

Suppose we have a full subcategory $\widetilde{\Cc}$ of $\Cc$ on which $F$ \emph{does} preserve weak equivalences (that is, sends $V$ to $W$) and such that $\widetilde{\Cc}[(W \cap \widetilde{\Cc})^{-1}] \cong \Cc[W^{-1}]$. So you add some isomorphisms and everything in $\Cc$ becomes isomorphic to one of the objects in $\widetilde{\Cc}$. Then we can define
\begin{center}
	\begin{tikzcd}
		\Cc \arrow[r,"F|_{\widetilde{\Cc}}"] \arrow[d,"L"] & \mathcal{D} \arrow[d,"L'"]\\
		\widetilde{\Cc}[W^{-1}] \cong \Cc[W^{-1}] \arrow[r,dotted,"\overline{F}"] & \mathcal{D}[V^{-1}].
	\end{tikzcd}
\end{center}
and the arrow \emph{does} exist. 

\begin{theorem}\label{Right_Kan_Extension}
	Suppose we have $\Cc \xrightarrow{F} \mathcal{D}$ and $\widetilde{\Cc} \subseteq \Cc$ as above, a functor $Q: \Cc \to \widetilde{\Cc} \hookrightarrow \Cc$, and a natural transformation $\tau: Q \to \op{Id}_{\Cc}$ which consists objectwise of weak equivalences.
	Also suppose $W$ satisfies the 2-of-3 condition, i.e., if in a diagram
	\begin{center}
		\begin{tikzcd}
			A \arrow[r,"f"] \arrow[dr,swap,"g \circ f"] & B \arrow[d,"g"]\\
			& C
		\end{tikzcd}
	\end{center}
	two of the three are weak equivalences, then the third is a weak equivalence.
	\marginpar{This is harmless for weak equivalences - if two of them are isomorphisms then the third should be too. We can enhance $\Cc$ without changing the localization.}

	Then $Q$ preserves equivalences by 2-of-3, so $F \circ Q$ preserves equivalences. So we have the following diagram:
	\begin{center}
		\begin{tikzcd}
			\Cc \arrow[r,bend right,"Q"] \arrow[r,bend left,"\op{Id}"] \arrow[d,"L"] & \Cc \arrow[r,"F"] & \mathcal{D} \arrow[d,"L'"]\\
			\Cc[W^{-1}] \arrow[rr,dotted,"\op{Ran}_L(L' \circ F)"] & & \mathcal{D}[V^{-1}]
		\end{tikzcd}
	\end{center}
\end{theorem}

\begin{proof}
	Exercise. Only do this if you're fond of abstract nonsense.
\end{proof}

\begin{example}
	Consider the colimit functor $\mathbf{Top}^{diag} \xrightarrow{\op{colim}}\mathbf{Top}$. Fact: ths preserves weak equivalences of diagram of the form
	\begin{center}
		\begin{tikzcd}
			A \arrow[r,"g"] \arrow[d,"f"] & B\\
			C
		\end{tikzcd}
	\end{center}
	where $A,B,C$ are CW-complexes and $f$ and $g$ are cofibrations, i.e., are inclusions of subspaces such that around the image there is a neighborhood which retracts onto it. (This is the ``good inclusion'' of the previous lecture.)
\end{example}

\emph{Fact}: Every map in $\mathbf{Top}$ factors as a cofibration followed by a weak equivalence. Namely, denote by $Mf$ the mapping cylinder $Mf := (X \times I \cup Y)/(x,0) \sim f(x)$, which we intuit as the following construction.

\begin{figure}[h]
\centering{
\resizebox{75mm}{!}{\input{mapping_cylinder.pdf_tex}}
\caption{Top view.}
\label{fig:Mapping Cylinder}
}
\end{figure}

The first morphism is the inclusion into the top of the cylinder, which is clearly a cofibration, and the second is the retraction of the cylinder to the base, which is a weak equivalence.

Given this other fact, we have a recipe for the derived functor of pushout, called homotopy pushout. Given a pushout diagram
\begin{center}
$D=$
	\begin{tikzcd}
		A \arrow[r] \arrow[d] & B\\
		C
	\end{tikzcd}
\end{center}
we factorize it using the above construction to get a diagram
\begin{center}
	\begin{tikzcd}
		A \arrow[r, rightarrowtail] \arrow[d,rightarrowtail] & \widetilde{B} \arrow[r,"\sim"] \arrow[d] & B\\
		\widetilde{C} \arrow[d,swap,"\sim"] \arrow[r] & P\\
		C
	\end{tikzcd}
\end{center}
and call $P$ the homotopy pushout of the diagram $D$. The smaller pushout diagram is the image $Q(D)$ of $D$ under the replacement functor in Theorem \ref{Right_Kan_Extension}. 

Going back to the previous example of the failure of the pushout [\textbf{Take a photo of this picture?}]

What we did here was a right Kan extension for homotopy, which is a bit surprising.

Likewise for the derived limit we replace maps by \emph{fibrations}, i.e. we have a homotopy lifting property (see video).

There are factorization systems on many categories. This has been done in abstract homotopy theory. For example, on $\mathbf{Cat}$ with $W =$ category equivalences, cofibrations are functors which are injective on objects, and fibrations are functors which are $\Cc \to \mathcal{D}$ inducing surjections $\op{Iso}(\Cc) \to \op{Iso}(\mathcal{D})$ whose source lies in $\op{Iso}(\Cc)$. 

\begin{exercise}
	Find factorizations for these notions in the above style. Think again about the pushout 
		\begin{center}
		\begin{tikzcd}
			\prod_W (* \to *) \arrow[r] \arrow[d,hook] & \Cc \arrow[d,"L"]\\
			\prod_W (* \xrightarrow{\sim} *) \arrow[r] & \Cc[W^{-1}].
		\end{tikzcd}
	\end{center}
	We can factor the upper arrow so that it's actually an inclusion on objects and... that should tell us something.
\end{exercise}

\begin{example}[(Homotopy Quotients)]
	Let $G$ be a topological group. Then we can form the category $G-\op{Top}$ of topological spaces with a continuous $G$-action and $G$-equivariant continuous map. Let $V$ be weak equivalences of the underlying spaces (ignoring the $G$-action). We have the quotient functor
	\[
	G\op{-Top} \xrightarrow{-/G} \op{Top}.
	\]
	We know this can go bad morally in general. We can invert these $G$-equivalences:
	\begin{center}
	\begin{tikzcd}	
		G\op{-Top} \arrow[r,"-/G"] \arrow[d] & \op{Top}\arrow[d]\\
		G\op{-Top}[V^{-1}] & \op{Top}[W^{-1}]
	\end{tikzcd}
	\end{center}
	As before, quotienting out the $G$-action does not respect weak equivalences. For example, let $S^\infty = \op{colim}(S^0 \hookrightarrow S^1 \hookrightarrow S^2 \hookrightarrow \dots)$. (5*) The outcome will be a contractible space.

	$\Z/2$ acts on both spaces; on $S^\infty$ by multiplying coordinates by $\pm 1$. Then quotient is left exact (?) so it commutes with colimits (??) so
	\begin{align*}
	S^\infty/(\Z/2) &= \op{colim}(S^0/(\Z/2)) \hookrightarrow S^1/(\Z/2) \hookrightarrow \dots)\\
	&= \op{colim}(* \to \R\P^1 \to \R\P^2 \to \dots)\\
	&= \R\P^\infty
	\end{align*}
	But $\pi_1(\R\P^\infty) = \Z/2 \ne 0 = \pi_1(*) = \pi_1(*/(\Z/2)$, so these spaces are not weakly equivalent.

	So we have to derive the quotient. First we need a statement along the lines that any $G$-action can be replaced by a ``good'' $G$-action.

	\emph{Fact}: $-/G$ respects equivalences between spaces with \emph{free} $G$-action (it really throws everything around - if an element has a fixed point it has to be the identity). 

	So given any $G$-space, need to replace it by an equivalent space with a free $G$-action, then take the quotient.

	Consider $X$ as the constant simplicial object ($X$ at every level, all maps identity), and consider the simplicial space
	\[
	\dots \to G \times G \times G \times X \to G \times G \times X \to G \times X 
	\]
	(6*) Then $\op{hocolim}\widetilde{X_\bullet}$ is a good replacement. 
\end{example}
\end{document}